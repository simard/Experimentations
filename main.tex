\documentclass[a4paper,fleqn]{report}
	\usepackage[francais,pstricks]{Lapin}

	\newcommand{\PHANToM}{\Nom{PHANToM}\xspace}
	\newcommand{\Omni}{\Nom{\PHANToM Omni}\xspace}

	\title{Expérimentations}
	%\author{Jean \Nom{Simard}}
	\author{}
	\date{\today}
\begin{document}
	\maketitle
	\chapter{Recherche d'une structure}
		\section{Scénario}
			L'objectif de la tâche à réaliser sera de trouver des structures particulières sur une molécule.
			Pour atteindre cet objectif, le ou les sujets pourront manipuler la molécule dans son ensemble (outil \emph{grab}) et la manipuler atome par atome (outil \emph{tug}).
			Les structures à identifier seront indiquées à l'aide de photographies.
			Les photographies pourront être montrées sur l'écran d'un autre ordinateur par exemple.

			À chaque lancement d'une simulation, les sujets peuvent explorer la molécule à l'aide de l'outil \emph{grab}.
			Dès qu'ils estiment être prêts, la photographie est montrée et ils peuvent commencer à déformer la molécule pour trouver la structure en utilisant au choix les outils \emph{grab} et/ou \emph{tug}.
			Il est possible que la même structure soit plusieurs fois présente sur la même molécule.
			Dans ce cas, la première structure trouvée sera validée.

			Pour valider la recherche d'une structure, les deux sujets devront attraper un atome de cette structure.
			Cela devrait permettre de mesurer un temps total d'exécution entre le moment où la photographie est montrée et le moment où la structure est trouvée.

			Avant de commencer la tâche, les sujets devront au préalable choisir quelle main ils utiliseront avec quel outil.
			Sachant qu'il y a trois outils disponibles et deux sujets, un des deux sujets pourra (mais ce n'est pas obligé) utiliser ces deux mains avec deux outils (un \emph{tug} et un \emph{grab} ou deux \emph{tug}).
			Afin de pouvoir faire ce choix d'association main/outil, une phase d'apprentissage (sur la molécule \Nom{trp-cage}) et de découverte des outils laissera une totale liberté pour la manipulation.

		\section{Constantes}
			Les constantes sont les invariants de l'environnement d'expérimentation.
			\begin{description}
				\item[Visuel] Un affichage unique pour tous les sujets par vidéo-projection.
				\item[Audio] Aucun retour audio pour ne pas perturber la communication orale entre les sujets.
				\item[Haptique] Haptique activé pour les interfaces de type \emph{tug} et désactivé pour les interfaces de type \emph{grab}.
				\item[Interface haptique] 2~interfaces \Omni avec l'option \emph{tug} et 1~interface \Omni avec l'option \emph{grab}.
				\item[Association main/interface] Pendant l'expérimentation, chaque utilisateur doit toujours utiliser une interface/outil avec la même main.
				Seule la phase d'apprentissage devra lui permettre de choisir avec quelle main il sera le plus à l'aise.
				\item[Positionnement] Les sujets sont faces à l'écran et côte à côte dans le cas d'une collaboration.
			\end{description}

		\section{Variables}
			Les variables de l'expérimentation.
			\begin{description}
				\item[Nombre de personnes] 1~ou 2~personnes;
				\item[Molécule] Variation sur la taille de la molécule avec \Nom{trp-zipper} puis \Nom{Prion}.
				\item[Orientation] L'état initial (orientation dans ce cas) de la molécule sera changé à chaque fois.
			\end{description}
			On établi ainsi quatre cas différents, l'orientation étant un facteur secondaire destiné à être une variation sur les multiples essais pour une configuration donnée.
			Pour chaque configuration, on fera passer 5~fois (avec 5~orientations différentes).

		\section{Enregistrements}
			Les enregistrements sont les valeurs à conserver pendant la durée de l'expérimentation.
			\begin{description}
				\item[Position] La position des effecteurs de toutes les interfaces haptiques à chaque instant.
				\item[Force] La force exercée sur les interfaces haptiques par les sujets à chaque instant.
				\item[Temps] À chaque position ou force enregistrée, on associe le temps courant.
				\item[Temps total] Le temps total de la réalisation de la tâche.
				\item[Audio] Enregistrement audio pendant toute la durée de la réalisation de la tâche.
				\item[Sujet] Différentes informations concernant chaque sujet sont nécessaires:
				\begin{enumerate}
					\item Prénom \Nom{Nom}
					\item Sexe
					\item Main dominante
					\item Affinités avec le collaborateur durant l'expérimentation
					\item Quelle main (gauche ou droite) utilise quelle outil (\emph{tug} ou \emph{grab}) ?
				\end{enumerate}
			\end{description}

		\section{Analyse}
			\subsection{Positions}
				Deux principales tendances seront étudiées.
				La première concernera les mouvements individuels des sujets.
				La seconde étudiera plus précisément les mouvements comparés des deux sujets.
				\subsubsection{Mouvements individuels}
					Pour les mouvements individuels, on étudiera la moyenne des positions ainsi que l'écart-type ce qui devrait nous permettre de décider une sphère moyenne d'évolution en \Nom{3d}.
					La moyenne donnera le centre de cette sphère et l'écart-type le rayon.

				\subsubsection{Mouvements comparés}
					Pour comparer les mouvements des deux sujets, deux opérations peuvent être effectuées.
					Tout d'abord, en utilisant l'analyse précédente, on devrait pouvoir dessiner deux sphères dans une même scène \Nom{3d} pour permettre de comparer les \emph{sphères} d'évolution des deux sujets.
					Cet affichage va mettre en évidence les recouvrements de zones de travail ou non et de déterminer la tendance des sujets à travailler ensemble dans une même zone ou à travailler ensemble mais sans se perturber.

					La seconde analyse qui pourra être faite sera de dessiner l'évolution de la distance entre les deux effecteurs.
					Cette information devrait recouvrir en partie les déductions faites à partir des sphères.

			\subsection{Vitesses}
				Les points suivants seront étudiés:
				\begin{enumerate}
					\item Vitesse moyenne;
					\item Écart-type.
				\end{enumerate}

			\subsection{Forces}
				Les points suivants seront étudiés:
				\begin{enumerate}
					\item Vitesse moyenne;
					\item Écart-type.
				\end{enumerate}

			\subsection{Audio}
				Concernant les enregistrements audio, c'est le temps de parole qui sera étudié.
				La donnée principale sera donc le temps de parole sur toute la durée de la tâche à réaliser.
				Pour cela, quelques contraintes:
				\begin{enumerate}
					\item Le début de la mesure commence lorsque la structure particulière à rechercher est montrée;
					\item Un silence de moins de \nombre{0,5}~secondes ne sera pas considéré comme un silence.
				\end{enumerate}
				Le temps de parole devrait nous permettre d'observer les points suivants:
				\begin{enumerate}
					\item Le pourcentage de parole pendant la tâche;
					\item Le rapport entre le pourcentage de parole pendant la tâche et le temps de réalisation de la tâche (efficience)\footnote{On ne parle pas d'efficacité ici car il est impossible d'évaluer la qualité de la solution. L'efficacité est une combinaison de l'efficience et de la qualité};
					\item Le pourcentage comparé de temps de parole de chacun des sujets.
				\end{enumerate}
\end{document}
